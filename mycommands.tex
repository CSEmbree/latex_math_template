\newenvironment{problem}[2][Problem]{\begin{trivlist}\item[\hskip \labelsep {\bfseries #1}\hskip \labelsep {\bfseries #2}]\itshape}{\end{trivlist}}
\usepackage{amsfonts,amssymb,amsmath,enumerate,ifthen}
\newcommand{\clos}[1]{\overline{#1}}
\newcommand{\inte}[1]{#1^O}
\newcommand{\proof}{ \noindent \textbf{Proof:} \ \ }
\newcommand{\solution}{ \noindent \textbf{Solution:} \ \ }
\newcommand{\proofends}{\ \,\\ \noindent\makebox[\linewidth]{\hfill \resizebox{0.35\linewidth}{1.2pt}{$\Box$}\hfill $\Box$ }\bigskip \bigskip}
\newcommand{\solutionends}{\ \,\\ \noindent\makebox[\linewidth]{\hfill \resizebox{0.35\linewidth}{1.2pt}{$\Box$}\hfill }\bigskip \bigskip}
 %\noindent\makebox[\linewidth]{\rule{.75\paperwidth}{.2pt}}


 % These are shortcuts for limits
     \newcommand{\limn}{\lim_{n\to\infty}}

 % These are shortcuts for integrals
     \newcommand{\fmu}{\int f\, d\mu}
     \newcommand{\fnmu}{\int f_n\, d\mu}
     \newcommand{\fla}{\int f\, d\lambda}
     \newcommand{\fnla}{\int f_n\, d\lambda}

 % These are shortcuts for other stuff
     \usepackage{amsfonts,ifthen,epsfig,amssymb,amsmath,color,graphicx,tikz}
     \newcommand{\mybox}[1]{\mbox{\ #1\ }}
     \newcommand{\unionn}{\bigcup_{n=1}^\infty}
     \newcommand{\intern}{\bigcap_{n=1}^\infty}
     \newcommand{\ds}[1]{{\displaystyle #1}}
     \newcommand{\at}[2][rrrrrrrrrrrrrr]{    %Se debe utilizar así \at[llll]{1&2&3\\ 4&5&5}
                                        % Cuando el argumento es opcional se ponen []
                     \begin{array}{#1}
                     #2\\
                     \end{array}}
    \newcommand{\defi}[2]{#1 = \left\{\at[lllllllllllll]{#2}\right.} %\defi{f(x)}{3 & x\le 4\\ 2 & x>4}
    \newlength{\tamano} %asignar una longitud de nombre tamano
    \newcommand{\linfantasma}[1]{\settowidth{\tamano}{#1}
                       \underline{\makebox[\tamano]{}}}
    \newcommand{\cua}[1]{\settowidth{\tamano}{#1}
                       \framebox[\tamano][c]{#1}}
                       
    \newcommand{\mynotes}{\vspace{2.5cm}{\small
      \begin{center}
      \begin{tabular}{ll}  \hline
        \multicolumn{2}{|c|}{\textsc{My Grading Style}}\\ \hline \\[-1mm]
        $\ \ \ \ \ \ \bigcirc$ & Fully Commented \\[1mm]
        $\ \ \ \ \ \ \bigcirc$ & Office Review\\[1mm]
        $\ \ \ \ \ \ \bigcirc$ & Total Anarchist
      \end{tabular}\\[1.5cm]
      \begin{tabular}{ll}   \hline
        \multicolumn{2}{|c|}{\textsc{My Late Penalty}}\\ \hline\\[1mm]
        Due Date       & \linfantasma{February 200000} \\[4mm]
        Day Turned in & \linfantasma{February 200000} \\[4mm]
        Late Penalty  & \linfantasma{February 200000}\\
      \end{tabular}\\[1.5cm]
      \cua{\
      \begin{tabular}{ll}
        \multicolumn{2}{c}{\tiny For Official Use Only}\\[1mm] \hline\\[0mm]
        \multicolumn{2}{c}{\textsc{Score}}\\[2mm]
        Total Points       & \linfantasma{2000000000} \\[1mm]
        Late Penalty    & \linfantasma{2000000000} \\[1mm]
        Final Score    & \linfantasma{2000000000} \\[1mm]
      \end{tabular}
      \ }
      \end{center}
      }}

     \newcommand{\mystart}{\date{}\maketitle\thispagestyle{empty}\vspace{1cm}\mynotes\newpage}
     \author{\,\\[1mm] \LARGE \textbf{\myhomework}\\[2mm]\underline{\myname}\\[4mm]{\LARGE \textsc{\mytopic}}}
     \newcommand{\finv}[1]{f^{-1}\left(#1\right)}

 % This is a shortcuts for indicator
     \DeclareSymbolFont{mine}{OML}{cmm}{m}{it}\DeclareMathSymbol{\chii}{0}{mine}{'037}
     \newcommand{\indic}[1]{\mathrel{\scalebox{1.25}{$\chii$}_{\raisebox{-.75ex}{\scriptsize{$#1$}}}}\hspace{-1.5mm}}

 % These are shortcuts for Reals, Rationals, etc.
     \newcommand{\R}{{\mathbb R}}
     \newcommand{\I}{{\mathbb I}}
     \renewcommand{\P}{{\mathbb P}}
     \newcommand{\Q}{{\mathbb Q}}
     \newcommand{\N}{{\mathbb N}}
     \newcommand{\C}{{\mathbb C}}
     \newcommand{\F}{{\mathcal F}}
     \newcommand{\Z}{{\mathbb Z}}
     \newcommand{\A}{\ensuremath{{\Bbb A}}}
     \newcommand{\B}{\ensuremath{{\Bbb B}}}
     \newcommand{\then}{\Rightarrow}


\newcommand{\fancihomework}[3]{
            \setlength{\topmargin}{1cm} %It was -2.5 for early versions of miktex (<2.8)
            \setlength{\headsep}{0.5in}
            \setlength{\textheight}{22.5cm}\setlength{\oddsidemargin}{0cm}\setlength{\evensidemargin}{0cm}\setlength{\textwidth}{15cm}
            \setlength{\headheight}{.5cm}
            \thispagestyle{empty}\usepackage{fancyhdr}
            \pagestyle{fancy}
            \fancyhf{} %delete the current section for header and footer
            \fancyhead[LE,RO]{\bfseries #3/\ \ \ \thepage}
            \fancyhead[LO,CE]{\bfseries #1}
            \fancyhead[CO,RE]{\nouppercase{\bfseries #2} }
            \renewcommand{\headrulewidth}{.5pt}
            \fancypagestyle{plain}{}}


\newcommand{\mini}[2]{\begin{minipage}{#1\textwidth}
            \begin{flushleft}#2\end{flushleft}\end{minipage}}
\newif \ifhint
\newcommand{\hint}[1]{\ifhint{{\tiny \textbf{Hint}: #1}}\else{}\fi}


\newcommand{\Union}[3]{\bigcup_{#3} #1_{#2}} %Usage \Union{E}{x}{y}= big union over all y of E_x
\newcommand{\Inter}[3]{\bigcap_{#3} #1_{#2}} %Usage \Inter{E}{x}{y}=big intersection over all y of E_x
\newcommand{\Aunion}[2]{\Union{#1}{#2}{\al\in \A}} %Usage \AUnion{E}{x}= big union over all \al in A of E_x
\newcommand{\Ainter}[2]{\Inter{#1}{#2}{\al\in \A}} %Usage \AInter{E}{x}= big intersection over all \al in A of E_x
\newcommand{\Ual}[1][E]{\Aunion{E}{\al}} %Usage \Ual{B}= big union over all \al in A of B_\al
                                         %Usage \Ual=big union over all \al in A of E_\al
\newcommand{\Ial}[1][E]{\Ainter{E}{\al}} %Usage \Ial{B}= big intersection over all \al in A of B_x
                                         %Usage \Ial=big intersection over all \al in A of E_\al
\newcommand{\UEl}[1][E_n]{\bigcup_{n=1}^\infty #1}
\newcommand{\IEl}[1][E_n]{\bigcap_{n=1}^\infty #1}
\newcommand{\Un}[1][E_n]{\bigcup_{n=1}^\infty #1}
\newcommand{\In}[1][E_n]{\bigcap_{n=1}^\infty #1}
\newcommand{\UE}[1][E_n]{\bigcup_{n=1}^\infty #1}
\newcommand{\IE}[1][E_n]{\bigcap_{n=1}^\infty #1}

\newcommand{\al}{\alpha}
\newcommand{\be}{\beta}
\newcommand{\ex}{\exists}
\newcommand{\fa}{\forall}

\newcommand{\textito}[1]{\hspace{-1.2cm}\textbf{#1}} 